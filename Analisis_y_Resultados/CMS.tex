 Este experimento consiste de varios subsistemas los cuales estan disenados para la identifcaci ́on de prácticamente todas las partículas del modelo estándar, para su dise\~no se tomó en cuenta como cada partícula interacciona con la materia, por ejemplo las part ́ıculas cargadas son identificadas por medio de detectores a base de silicio y degas noble los cuales permiten determinar con precisi ́on el tiempo de cruce y localizaci ́onespacial de las part ́ıculas, adem ́as de que el signo de la carga es determinado deacuerdoa la deflecci ́on de su trayectoria debido al poderoso campo magn ́etico solenoide de 4Teslas que envuelve al CMS. Las part ́ıculas neutras son identificadas por la energ ́ıa quedepositan en los calorimetros, la variedad de interacciones por tipo de part ́ıcula se puedever en la Figura 1.3. Los muones son part ́ıculas que interaccionan d ́ebilmente con lamateria por lo que su detecci ́on se da un dos subsistemas, el detector de trazas, quecorresponde a la primera capa de CMS y el sistema de muones, ultima capa del detector,lo cual permite una reconstrucci ́on de trayectoria muy precisa, debido a esto el posibledecaimiento de las nuevas part ́ıculas a muones resulta un canal favorecido desde el puntode vista experimenta