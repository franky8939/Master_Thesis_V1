
En el año 1973 por Julius Wess y Bruno Zumino presentan un modelo en la física de partículas el cual es conocido con el nombre de Modelo de Wess-Zumino, este es un modelo mínimo supersimétrico con solo un Fermion y su super compañero Boson. A pesar de que el modelo de
Wess-Zumino no representa un modelo físico real, sirve por su sencillez de modelo ejemplo para mostrar ciertos aspectos de los modelos físicos supersimétricos. El primer modelo supersimétrico compatible con el modelo estandar de la física de partículas llamado \MSSM(\textbf{M}odelo \textbf{M}ínimo \textbf{E}stándar \textbf{S}upersimétrico) este fue enunciado en el año 1981 por Howard Georgi y Savas Dimopoulos. Según el \MSSM, las masas de los super compañeros se podrán observar en la región entre $100~GeV$ hasta $1~TeV$ mediante un acelerador de partículas, terminado de construir en el año 2008 en la frontera
franco-suiza. Los científicos esperan poder demostrar mediante el \LHC~ la existencia de los super compañeros de las partículas elementales ya conocidas.