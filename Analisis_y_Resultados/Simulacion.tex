
Gracias a la infraestructura desarrollada dentro del experimento \textbf{ATLAS} y el \CERN, los equipos de análisis de fisica de altas energías ahora pueden preservar fácilmente su código de análisis en formatos de contenedores linux, de modo que pueda usarse con fines de reinterpretación, con ellos viene incluido como una receta, el orden exacto en que las diversas tareas de un análisis deben llevarse a cabo y el conocimiento de cómo usarlo exactamente para poder extraer nueva ciencia.

Entre las herramientas más básicas y robusta de la biblioteca desarrollada por el \CERN ~es el programa orientado a objetos \textbf{ROOT}, este fue originalmente fue diseñado para el análisis de datos de física de partículas y contiene varias características específicas de este campo. Este proporciona todas las funcionalidades necesarias para manejar el procesamiento de grandes datos, el análisis estadístico, la visualización y el almacenamiento. Está escrito principalmente en $C++$ pero integrado con otros lenguajes como Python y R, es la base también de muchos de sus sistemas, conteniendo las librerias necesarias para su ejecución.

El proyecto \textbf{RECAST} (\textbf{R}equest \textbf{E}fficiency \textbf{C}omputation for \textbf{A}lternative \textbf{S}ignal \textbf{T}heories) combina la motivación científica para un poderoso programa de reinterpretación en el \LHC ~ con las capacidades técnicas que ofrecen los lenguajes de flujo de trabajo y los entornos de software preservables. Los principales grupos de búsqueda dentro de la colaboración \LHC~ ahora requieren que se conserven nuevos análisis utilizando estas nuevas herramientas, de modo que cuando los teóricos proponen un nuevo modelo de física, la colaboración puede reutilizar estos análisis archivados para derivar una primera evaluación a través de la reinterpretación. También se espera que los análisis conservados se usen en una ola de estudios de resumen planificados una vez que se finalicen los análisis de datos de la segunda ejecución del \LHC, entre ellos los modelos supersimétricos, denominado \textbf{MSSM} fenomenológico y de esta forma permitir una evaluación detallada del estado de la supersimetría más allá del alcance más estrecho de los modelos individuales. 

La implementación de estas herramientas y su completo control son habilidades necesarias para incursionar en la investigación de altas energías, de aquí que se precise profundizar en ellas.
