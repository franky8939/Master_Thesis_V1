\documentclass[11pt,english]{report}

\usepackage{paquetes_todos}

\usepackage[showboxes]{textpos}
%\setlength{\baselineskip}{1}

\newenvironment{itemize_f}%
  {\begin{itemize}%
    \setlength{\itemsep}{0pt}%
    \setlength{\parskip}{0pt}}%
  {\end{itemize}}
%\setlength\itemsep{-.2cm} %
\usepackage{framed}

\usepackage{lineno}
\linenumbers
\usepackage{natbib}
\usepackage{dsfont}
\usepackage{multibib}

\begin{document}
% Portada
\title{Activity report Thesis}
\author{Francisco Martínez Sánchez}
\prevdegrees{Bs. Physics}
%\prevdegrees{Grados previos}
\institute{UNISON}
\department{DIFUS}
\degree{Master}
\supervisor{Dr. Alfredo Castañeda}
\city{Sonora/Hermosillo}
\degreemonth{Mes}
\degreeyear{Año}
\maketitle
%\begin{acknowledgements}

%\end{acknowledgements}

\tableofcontents
\clearpage
\listoftables
\clearpage
\listoffigures
\clearpage

\begin{abstract}
\end{abstract}

%\begin{summary}
%\end{summary}

\pagenumbering{arabic}



\input{link.tex}
%overleaf
%local
\input{Introduccion/Introduccion.tex}

%%%%%%%%%%%%%%%%%%%%%%%%%%%%%%%
%% Cap Fisica de Particulas %%
%%%%%%%%%%%%%%%%%%%%%%%%%%%%%%%

\chapter{Física de Partículas}
\input{Fisica_de_Particulas/Introduccion.tex}

	\section{El Universo y su evolución} 
	\input{Fisica_de_Particulas/Universo.tex}
	
	\section{Modelo Estándar} 	
	\input{Fisica_de_Particulas/Modelo_estandar.tex}
	
		\subsection{Composición de la Materia y Fuerzas Fundamentales} 
		\input{Fisica_de_Particulas/Materia_y_Fuerzas.tex}
		
		\subsection{Simetrías y Lagrangiano*} 
		\input{Fisica_de_Particulas/lagrangiano_modelo_estandar.tex}
	
		\subsection{Insuficiencias del modelo*}
		Incluso cuando el \ME ha tenido gran éxito en explicar los resultados experimentales, tiene ciertas cuestiones importantes sin resolver:

\begin{itemize}
\item[-] 

\item[-] 

\item[-] 

\item[-] 

\item[-] 

\item[-] 

\item[-] 

\item[-] 

\item[-] 
\end{itemize}





	
	
	\section{M\'as all\'a del modelo est\'andar con la Materia Oscura}
	\input{Fisica_de_Particulas/Materia_Oscura.tex}
	
		\subsection{Evidencias observacionales}
		\input{Fisica_de_Particulas/Pruebas_Oscuras.tex}
		
		\subsection{Alternativas Teóricas}
		\input{Fisica_de_Particulas/Alternativas_Oscura.tex}
		
		\subsection{Composición de la Materia Oscura}
		\input{Fisica_de_Particulas/Materia_Oscura_Composicion.tex}
		
		\subsection{Experimentos de la Materia Oscura}
		\input{Fisica_de_Particulas/Experimentos_Oscura.tex}		

%%%%%%%%%%%%%%%%%%%%%%%%%%%%%%%
%% Cap Antecedentes Teóricos %%
%%%%%%%%%%%%%%%%%%%%%%%%%%%%%%%

\chapter{Antecedentes Teóricos y Experimentales.}
\input{Analisis_y_Resultados/Introduccion.tex}
		
	\section{Física de Altas Energías.}
	La Organización Europea para la Investigación Nuclearo \CERN (\textbf{C}onseil \textbf{E}uropéen pour la \textbf{R}echerche \textbf{N}ucléaire) es una organización de investigación europea que opera el laboratorio de física de partículas más grande del mundo, está situado en Suiza cerca a la frontera con Francia, entre la comuna de Saint-Genis-Pouilly y la comuna de Meyrin. La función principal del \CERN ~ es proporcionar los aceleradores de partículas y otra infraestructura necesaria para la investigación de física de alta energía; como resultado, se han construido numerosos experimentos en el \CERN ~ a través de colaboraciones internacionales. El sitio principal de Meyrin alberga una gran instalación informática, que se utiliza principalmente para almacenar y analizar datos de experimentos, así como para simular eventos. Los investigadores necesitan acceso remoto a estas instalaciones, por lo que el laboratorio ha sido históricamente un importante centro de red de área amplia. En la Fig. \ref{cern}a se muestra un diagrama de las instalaciones y los proyectos en los que está dividido. 

El \CERN ~ es fundamentalmente un conjunto interconectado de aceleradores de partículas cuyo primer elemento, el Sincro-Ciclotrón de protones de $600 ~ MeV$ o \textbf{SC} (\textbf{S}ynchro-\textbf{C}yclotron) se empiezó a construir a mediados de 1955, sustituido por el Gran Coalisión de Hadrones  o \LHC (\textbf{L}arge \textbf{H}adron \textbf{C}ollider) puesto en funcionamiento el 2008. En la actualidad, gran parte de la actividad experimental que se realiza en el \CERN ~ está concentrada en la construcción de los experimentos para el \LHC:
\begin{itemize_f}
\item[-] \textbf{ATLAS} (\textbf{A T}oroidal \textbf{L}HC \textbf{A}pparatu\textbf{S}) : Investiga una amplia gama de física, desde la búsqueda del bosón de Higgs hasta dimensiones adicionales y partículas que podrían formar materia oscura. Aunque tiene los mismos objetivos científicos que el experimento \CMS, utiliza diferentes soluciones técnicas y un diseño de sistema magnético diferente.

\textbf{Página del proyecto :} \url{https://atlas.cern/}%\url{https://home.cern/science/experiments/atlas} %~~~~~~~~~~~~~~~~~~~~~~~~~~~~~~~~~

\item[-] \CMS \href{https://en.wikipedia.org/wiki/Compact_Muon_Solenoid}{(\textbf{C}ompact \textbf{M}uon \textbf{S}olenoid) :} Tiene un amplio programa de física que va desde el estudio del Modelo estándar (incluido el bosón de Higgs) hasta la búsqueda de dimensiones y partículas adicionales que podrían formar materia oscura. Está construido alrededor de un gran imán de solenoide.

\textbf{Página del proyecto :} \url{https://cms.cern/detector}%\url{https://home.cern/science/experiments/cms} %~~~~~~~~~~~~~~~~~~~~~~~~~~~~~~~~~

\item[-] \href{https://es.wikipedia.org/wiki/LHCb}{\textbf{LHCb} (\textbf{L}arge \textbf{H}adron \textbf{C}ollider \textbf{b}eauty) :} experimento especializado en física del \quark ~ b, algunos de cuyos objetivos son la medida de parámetros de violación de simetría \textbf{CP} en las desintegraciones de hadrones que contengan dicho \quark ~ o la medida de precisión de las fracciones de desintegración (``branching ratios'') de algunos procesos extremadamente infrecuentes.

\textbf{Página del proyecto :} \url{http://lhcb-public.web.cern.ch/lhcb-public/}

\item[-] \href{https://en.wikipedia.org/wiki/ALICE_experiment}{\textbf{ALICE} (\textbf{A L}arge \textbf{I}on \textbf{C}ollider \textbf{E}xperiment) :} es un detector de iones pesados, estudiar la física de la materia que interactúa fuertemente a densidades de energía extremas, donde se forma una fase de la materia llamada plasma quark-gluón. 

\textbf{Página del proyecto :} \url{http://aliceinfo.cern.ch/Public/Welcome.html}

%\item[-] \href{https://en.wikipedia.org/wiki/Proton_Synchrotron}{\textbf{PS} (\textbf{P}roton \textbf{S}ynchrotron) :} es un componente clave en el complejo acelerador del \CERN, donde generalmente acelera los protones suministrados por el Proton Synchrotron Booster o los iones pesados del Anillo de iones de baja energía (LEIR). En el curso de su historia, ha hecho malabarismos con muchos tipos diferentes de partículas, alimentándolas directamente a experimentos o aceleradores más potentes.

%\item[-] SPS
\end{itemize_f}

\begin{figure}[h!]
    \centering
    \includegraphics[width=1\textwidth]{Analisis_y_Resultados/imagenes/cern.png}
    \caption{(a) Diagrama de los experimentos que componen el centro de investigación del \CERN. (b) Imagenes del experimento \LHC. Adaptada de la página: \url{https://theconversation.com/goodbye-for-a-while-to-the-large-hadron-collider-12238}.}
    \label{cern}
\end{figure}

Uno de los experimentos considerado por sus resultados de los mas importantes es el \CMS, el cual es uno de los detectores multi-usos del \CERN ~ como se puedo constatar anteriormente, dicho detector tiene la capacidad de cubrir un amplio rango de procesos físicos, siendo este junto con el experimento \textbf{ATLAS} los que reportaron la observación de la partícula de Higgs en el 2012. El mismo es uno de los recursos principales para las investigaciones relacionadas con la exploración de la materia oscura.

El experimento \LHC ~está continuamente en proceso de actualización con el objetivo de proporcionar mediciones más precisas de nuevas partículas y permitiendo observar raros procesos teorizados y de esta intentar aumentar  nuestros conocimientos de la materia oscura. Con la detección de eventos raros aumentaría nuestra comprensión de la frontera energética y arrojaria luz sobre la teoria de la materia oscura. Las intenciones en el núcleo del proyecto \LHC ~ es con procesos de alta luminosidad (\textbf{HL-}\LHC), cuya fase de diseño fue apoyada en parte por fondos del Séptimo Programa Marco de la Comisión Europea y está actualmente en desarrollo.







		
		\subsection{Actualizando HLC.}	
		\input{Analisis_y_Resultados/LHC_Actualizacion.tex}
	
	\section{Experimento CMS.}
	\input{Analisis_y_Resultados/Experimento_CMS_del_CERN.tex}
	
	
		\subsection{Reconstruccion de Muones.}
		\input{Analisis_y_Resultados/Reconstruccion_Muones.tex}
		\subsection{Actualizando CMS.*}	
		\input{Analisis_y_Resultados/CMS_Actualizacion.tex}
		
	\section{Simulación en Altas Energías.}
	\input{Analisis_y_Resultados/Simulacion.tex}
	
		\subsection{Implementando ROOT.}
		\input{Analisis_y_Resultados/Root.tex}
		
		\subsection{Visualización de eventos con EVE.}
		\input{Analisis_y_Resultados/EVE.tex}
		
		\subsection{Altas Energías con MadGraph.}
		\input{Analisis_y_Resultados/MadGraph.tex}

		\subsection{Hadronizacion con Pythia 8.}
		\input{Analisis_y_Resultados/Pythia.tex}

		\subsection{Simulando el detector con Delphes 3.}
		\input{Analisis_y_Resultados/Delphes.tex}

		

	\section{Extensión del Modelo Estándar con Supersimetría.*}
	
En el año 1973 por Julius Wess y Bruno Zumino presentan un modelo en la física de partículas el cual es conocido con el nombre de Modelo de Wess-Zumino, este es un modelo mínimo supersimétrico con solo un Fermion y su super compañero Boson. A pesar de que el modelo de
Wess-Zumino no representa un modelo físico real, sirve por su sencillez de modelo ejemplo para mostrar ciertos aspectos de los modelos físicos supersimétricos. El primer modelo supersimétrico compatible con el modelo estandar de la física de partículas llamado \MSSM(\textbf{M}odelo \textbf{M}ínimo \textbf{E}stándar \textbf{S}upersimétrico) este fue enunciado en el año 1981 por Howard Georgi y Savas Dimopoulos. Según el \MSSM, las masas de los super compañeros se podrán observar en la región entre $100~GeV$ hasta $1~TeV$ mediante un acelerador de partículas, terminado de construir en el año 2008 en la frontera
franco-suiza. Los científicos esperan poder demostrar mediante el \LHC~ la existencia de los super compañeros de las partículas elementales ya conocidas.
	
		\subsection{Supersimetria.*}
		En la física de partículas, la supersimetría es una simetría hipotética propuesta querelacionaría las propiedades de los bosones y los fermiones. Aunque todavía no se haverificado experimentalmente que la supersimetría sea una simetría de la naturaleza, es parte fundamental de muchos modelos teóricos, incluyendo la teoría de supercuerdas, la
supersimetría también esconocida por \SUSY (\textbf{SU}per\textbf{SY}mmetry) siendo una de las teorías más populares que postulan la existencia de física más allá del \ME. 
\begin{figure}[h!]
\centering
\includegraphics[width=1\textwidth]{Analisis_y_Resultados/imagenes/supersimetrias.png}
\caption{Extensión del Modelo Estandar bajo la existencia de la supersimetría (\SUSY). Página de origen : \href{http://www.cienciakanija.com/2009/11/13/confiamos-en-susy-lo-que-realmente-busca-el-lhc/}{\texttt{http://\-www.cienciakanija.com/\-2009/\-11/13/\-con\-fia\-mos-\-en-\-susy-\-lo-\-que-\-real\-men\-te-bus\-ca-el-lhc/}}.}
\label{susy}
\end{figure}

De forma general el \ME ~se construye a partir de simetrías muy fundamentales que dan lugar a leyes de conservación, en el caso de \SUSY, esta incluye todas las simetrías que ya contiene el \ME~ y añade otra más que involucra al espín. Lo que postula \SUSY~ es que a cada partícula del \ME le corresponde un compañero supersimétrico que tiene el espín contrario, o sea, por cada fermión, \SUSY añade un bosón y por cada bosón añade un fermión. Por tanto, el número de partículas predicho por \SUSY es el doble que en el Modelo Estándar, como se visualiza en la Figura~\ref{susy}.

Se teoriza que \SUSY puede dar solución al problema de la materia oscura mediante su teorizada relación con la materia del \ME, en la mayoría de modelos de supersimetría, la partícula supersimétrica más ligera o \LSP (mencionada anteriormente) es necesariamente neutra y estable. Esto significa que nuestro Universo estaría lleno de estas partículas masivas, neutras y estables, que por tanto serían buenas candidatas a formar la materia oscura.

Sin embargo, debido a que dichas compañeras supersimétricas aún no han podido ser creadas en el laboratorio, sus masas deben ser mucho mayores que las de las partículas originales. Esto implica que la supersimetría, de ser cierta,está rota por algún mecanismo, la especificación de dicho mecanismo da lugar a diversas simplificaciones del \MSSM, donde algunas partículas supersimétricas, como el neutralino, podrían explicar el problema de la materia oscura del universo.

%Con la verificación de la existencia de \SUSY se consiguiria medir la masa de la \LSP seremos capaces de decir mucho más sobre si con SUSY es suficiente para explicar la materia oscura o si se necesita algo más.

%{\let\clearpage\relax \chapter{bar}}
		
		\subsection{Modelo Minimo Estandar Simétrico.*}		
		\input{Analisis_y_Resultados/Dark_matter_model.tex}
		
	%\section{Antecedentes Experimentales del MSSM.}	
		
		
	
	%\section{Configuración de software para su implementación}
	%\input{Analisis_y_Resultados/Materiales_y_Metodos.tex}

%%%%%%%%%%%%%%%%%%%%%%%%%%
%% Other %%
%%%%%%%%%%%%%%%%%%%%%%%%%%
\chapter{Simulación y Análisis de Resultados.*}
%\input{Simulaciuon/Introduccion.tex}


%\section{Muon efficiency parametrization}

%\section{Integration of Dark-SUSY model}

%\section{Delphes Simulation}

%\section{Muon system}

%\subsection{CMS detector}

%\subsection{Luminosity Era}

%\chapter{Simulation}

%\section{Parton Level and Hadronization}

%\section{Detector simulation}

%\chapter{Next steps}

%% Resultados y Analisis
%\input{Analisis_y_Resultados/Introduccion.tex}

%\input{Analisis_y_Resultados/Analisis_y_Resultados.tex}

%%%%%%%%%%%%%%%%%%%%%%%%%%%%%%%%%%%%%%%%%%

%\bibliographystyle{UNAMThesis}
%\bibliographystyle{abbrv}
%\bibliographystyle{APA}
\bibliographystyle{UNAMThesis} % estilo bibliografico
\bibliography{ParticleUnison}
\addcontentsline{toc}{chapter}{Referencias Bibliográficas}
\end{document}