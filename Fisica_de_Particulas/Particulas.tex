Es de conocimiento general que los átomos tienen dos partes: la corteza en la que se encuentran los electrones y el núcleo que se compone de protones y neutrones. Los electrones tienen carga eléctrica negativa y son los responsables de, por ejemplo, conducir la electricidad (cuándo están libres) o hacer que las cosas tengan un color u otro (debido a los saltos que dan los electrones entre los diferentes niveles de energía posibles del átomo, pero esa es otra historia). Los protones tienen carga eléctrica positiva y en número es la misma que la de los electrones, por lo que tenemos que en condiciones normales el átomo es eléctricamente neutro. Los neutrones no tienen carga eléctrica, es decir, son neutros.