\begin{textblock}{9}(2.5,-4.5)
\begin{flushright}
\setlength{\baselineskip}{15pt}
\textblockrulecolour{white}
~

``No hay nada que hagan los seres vivos que no pueda entenderse desde el punto de vista de que están hechos de átomos que actúan de acuerdo con las leyes de la física.''\\[.5cm]
\textit{Richard P. Feynman}
\end{flushright}
\end{textblock}

%\framebox[0.8][r]{Bummer, I am too wide}


Encontrar los fundamentos del funcionamiento de los objetos materiales que componen la naturaleza ha sido una de las tareas de la que se ha ocupado la humanidad desde tiempos inmemorables, esta cuestión se abrió paso dentro de la química del siglo XIX comenzando con la existencia del átomo con Dalton (1803) y pasó a ser parte de la física con el descubrimiento del electrón por Thomson (1906) (teorizado por G. Johnstone Stoney (1881)) y de la radioactividad por el físico francés Antoine Henri Becquerel (1896).

Para los inicios del siglo XX es cuando el área de Física de Partículas Elementales se forma como área independiente junto con el establecimiento de la composición del núcleo atómico y con el advenimiento de los aceleradores, la misma se establece como la ciencia que estudia los componentes elementales de la materia y las interacciones entre ellos, también se la conoce como física de altas energías, con la cual se intenta teorizar sobre los origenes del comportamiento del universo. 






















