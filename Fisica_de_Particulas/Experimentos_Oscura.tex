Como ya se ha podido ver en las secciones anteriores, muchos fenónemos cosmólogicos han dado indicios de la existencia de materia oscura en sus diferentes composiciones teóricas, dado por lo cual un gran conjunto de experimentos han sido dedicados únicamente con la intención de obtener información pertinente en la comprensión de su composición y explicación de su comportamiento, existen dos métodos para realizar mediciones dimensionales:

% https://www.mpi-hd.mpg.de/hfm/CosmicRay/CosmicRaySites.html
\subsubsection{Forma directa}
Los métodos de detección directa intentan detectar las esporádicas interacciones que, a su paso por la Tierra, podrían experimentar
las partículas de materia oscura con un material adecuado y muy bien aislado del entorno.%Las mediciones directas utilizan instrumentos de medición para medir las propiedades físicas del objeto directamente, las mismas se pueden realizar en un amplio rango, especificado por la escala del instrumento de medición. 
Algunos experimentos de masa oscura son:

\begin{itemize_f}
\item[-] \href{https://en.wikipedia.org/wiki/Axion_Dark_Matter_Experiment}{ \textbf{ADMX} (\textbf{A}xion \textbf{D}ark \textbf{M}atter e\textbf{X}periment) :} 
\begin{itemize_f}
\item \textbf{Nombre:} \href{https://en.wikipedia.org/wiki/Axion_Dark_Matter_Experiment}{Experimento de Materia Oscura \Axion}
\item \textbf{Resumen:} Utiliza una cavidad de microondas resonante dentro de un gran imán superconductor para buscar \axiones~ de materia oscura fría en el halo local de materia oscura galáctica. 
\item \textbf{Pagina del proyecto :} \url{https://depts.washington.edu/admx/publications.shtml\#}
\end{itemize_f}




\item[-]\href{https://en.wikipedia.org/wiki/ANAIS}{\textbf{ANAIS} (\textbf{A}nnual modulation with \textbf{NAI} \textbf{S}cintillators) :} 
\begin{itemize_f}
\item \textbf{Nombre:} \href{https://en.wikipedia.org/wiki/ANAIS}{Modulación anual con NaI} \href{https://es.wikipedia.org/wiki/Centelleador}{Centelleador}.
\item \textbf{Resumen:} Busca la modulación anual de la señal con centelleadores de \href{https://es.wikipedia.org/wiki/Yoduro_de_sodio}{$NaI$} con el objetivo de detectar directamente  la Materia Oscura galáctica a través de su dispersión con los núcleos blanco de un cristal de NaI(Tl) radiopuro. Esta señal de Materia Oscura debería estar modulada anualmente debido al cambio de la velocidad relativa \WIMP-núcleo, consecuencia de la rotación de la Tierra alrededor del Sol.
\item \textbf{Pagina del proyecto :} \url{https://gifna.unizar.es/anais/}.
\end{itemize_f}

\item[-] \href{https://en.wikipedia.org/wiki/ArDM}{\textbf{ArDM} (\textbf{Ar}gon \textbf{D}ark \textbf{M}atter): }
\begin{itemize_f}
\item \textbf{Nombre:} \href{https://en.wikipedia.org/wiki/ArDM}{Materia Oscura en el Argón.}
\item \textbf{Resumen:} Busca medir y observando electrones libres de ionización y fotones de centelleo, que son producidos por la interación de su núcleo con los átomos vecinos y de esta forma relacionarla con la dispersión elástica de \WIMP ~ de los núcleos de argón líquido del que esta hecho el detector.
\item \textbf{Pagina del proyecto :} \url{https://wikimili.com/en/China_Jinping_Underground_Laboratory}.
\end{itemize_f}

%\item[-] \href{https://en.wikipedia.org/wiki/China_Dark_Matter_Experiment}{China Dark Matter Experiment (CDEX)}
%\begin{itemize_f}
%\item \textbf{Nombre:} 
%\item \textbf{Resumen:} 
%\item \textbf{Pagina del proyecto :} \url{}
%\end{itemize_f}

\item[-] \href{https://en.wikipedia.org/wiki/Cryogenic_Dark_Matter_Search}{\textbf{CDMS} (\textbf{C}ryogenic \textbf{D}ark \textbf{M}atter \textbf{S}earch)}
\begin{itemize_f}
\item \textbf{Nombre:} \href{https://en.wikipedia.org/wiki/Cryogenic_Dark_Matter_Search}{Buscando Materia Oscura Criogénica}
\item \textbf{Resumen:} Busca utilizando una serie de detectores de semiconductores a temperaturas de milikelvin encontrar los límites más sensibles en las interacciones de la materia oscura \WIMP~  con materiales terrestres y de esta manera detectar directamente la materia oscura. Constituye una serie de experimentos continuos: el \textbf{CDMS I}, \textbf{CDMS II}, el \textbf{SuperCDMS} y en la actualidad continúa con \textbf{SuperCDMS SNOLAB}.
\item \textbf{Pagina del proyecto :} \url{https://supercdms.slac.stanford.edu/}
\end{itemize_f}

%\item[-] \href{https://en.wikipedia.org/wiki/Cryogenic_Low-Energy_Astrophysics_with_Neon}{The Cryogenic Low-Energy Astrophysics with Noble liquids (CLEAN)}

%\item[-] \href{https://en.wikipedia.org/wiki/CoGeNT}{CoGeNT}
%\begin{itemize_f}
%\item \textbf{Nombre:} 
%\item \textbf{Resumen:} 
%\item \textbf{Pagina del proyecto :} \url{}
%\end{itemize_f}

%\item[-] \href{https://en.wikipedia.org/wiki/DAMA/LIBRA}{DAMA/LIBRA experiment}
%\begin{itemize_f}
%\item \textbf{Nombre:} Experimento \textbf{DAMA/LIBRA}
%\item \textbf{Resumen:} 
%\item \textbf{Pagina del proyecto :} \url{}
%\end{itemize_f}

\item[-] \href{https://en.wikipedia.org/wiki/DAMA/NaI}{DAMA/NaI experiment}
\begin{itemize_f}
\item \textbf{Nombre:} \href{https://en.wikipedia.org/wiki/DAMA/NaI}{Experimento DAMA/NaI}
\item \textbf{Resumen:} Características semejantes al experimento \href{https://en.wikipedia.org/wiki/ANAIS}{\textbf{ANAIS}} con mas de 7 años de datos de datos recopilados, fue continuado su estudio con el experimento \href{https://en.wikipedia.org/wiki/DAMA/LIBRA}{DAMA/LIBRA}.
\item \textbf{Pagina del proyecto :} \url{https://people.roma2.infn.it/~dama/web/home.html}
\end{itemize_f}

\item[-] \href{https://en.wikipedia.org/wiki/DarkSide}{DarkSide}
\begin{itemize_f}
\item \textbf{Nombre:} \href{https://en.wikipedia.org/wiki/DarkSide}{DarkSide}
\item \textbf{Resumen:} Busca con la construcción y operación de una serie de cámaras de proyección de tiempo o \href{https://en.wikipedia.org/wiki/Time_projection_chamber}{\textbf{TPC} (\textbf{T}ime \textbf{P}rojection \textbf{C}hamber)} de argón líquido para detectar \WIMPs. 
\item \textbf{Pagina del proyecto :} \url{http://darkside.lngs.infn.it/}
\end{itemize_f}

\item[-] \href{https://en.wikipedia.org/wiki/DEAP}{\textbf{DEAP} (\textbf{D}ark matter \textbf{E}xperiment using \textbf{A}rgon \textbf{P}ulse-shape discrimination)}
\begin{itemize_f}
\item \textbf{Nombre:} Experimento de materia oscura con discriminación de forma de pulso de argón
\item \textbf{Resumen:}) Busca discriminación de fondo basada en la característica forma de pulso de centelleo del argón permitiendo medir directamente \WIMP.
\item \textbf{Pagina del proyecto :} \url{http://deap3600.ca/}
\end{itemize_f}

%\item[-] \href{https://en.wikipedia.org/wiki/Monopole,_Astrophysics_and_Cosmic_Ray_Observatory}{MACRO (Monopole, Astrophysics and Cosmic Ray Observatory)}
%\begin{itemize_f}
%\item \textbf{Nombre:} 
%\item \textbf{Resumen:} 
%\item \textbf{Pagina del proyecto :} \url{}
%\end{itemize_f}
%
%\item[-] \href{https://en.wikipedia.org/wiki/PandaX}{PandaX (Particle and Astrophysical Xenon Detector)}
%\begin{itemize_f}
%\item \textbf{Nombre:} 
%\item \textbf{Resumen:} 
%\item \textbf{Pagina del proyecto :} \url{}
%\end{itemize_f}
%
%\item[-] \href{https://en.wikipedia.org/wiki/WIMP_Argon_Programme}{WIMP Argon Programme (WARP) }
%\begin{itemize_f}
%\item \textbf{Nombre:} 
%\item \textbf{Resumen:} 
%\item \textbf{Pagina del proyecto :} \url{}
%\end{itemize_f}
%
%\item[-] \href{https://en.wikipedia.org/wiki/XENON}{XENON}
%\begin{itemize_f}
%\item \textbf{Nombre:} 
%\item \textbf{Resumen:} 
%\item \textbf{Pagina del proyecto :} \url{}
%\end{itemize_f}
%
%\item[-] \href{https://en.wikipedia.org/wiki/ZEPLIN-III}{ZEPLIN-III dark matter experiment}
%\begin{itemize_f}
%\item \textbf{Nombre:} 
%\item \textbf{Resumen:} 
%\item \textbf{Pagina del proyecto :} \url{}
%\end{itemize_f}
%
%\item[-] \href{https://en.wikipedia.org/wiki/UK_Dark_Matter_Collaboration}{UK Dark Matter Collaboration (UKDMC)}
%\begin{itemize_f}
%\item \textbf{Nombre:} 
%\item \textbf{Resumen:} 
%\item \textbf{Pagina del proyecto :} \url{}
%\end{itemize_f}

\item[-] Otros experimentos : 
\begin{itemize_f}
\item \href{https://en.wikipedia.org/wiki/Monopole,_Astrophysics_and_Cosmic_Ray_Observatory}{\textbf{MACRO} (\textbf{M}onopole, \textbf{A}strophysics and \textbf{C}osmic \textbf{R}ay \textbf{O}bservatory)}, 

\textbf{Pagina del proyecto :} \url{https://hepwww.pp.rl.ac.uk/groups/ukdmc/ukdmc_old.html}

\item \href{https://en.wikipedia.org/wiki/PandaX}{PandaX (\textbf{P}article and Astrophysical \textbf{X}enon Detector)}, 

\textbf{Pagina del proyecto :} \url{https://pandax.sjtu.edu.cn/}

\item \href{https://en.wikipedia.org/wiki/WIMP_Argon_Programme}{\textbf{WARP} (\textbf{W}IMP \textbf{AR}gon \textbf{P}rogramme)}, 

\textbf{Pagina del proyecto :} \url{https://ztopics.com/WIMP\%20Argon\%20Programme/}

\item \href{https://en.wikipedia.org/wiki/XENON}{XENON}, 

\textbf{Pagina del proyecto :} \url{http://www.xenon1t.org/}

\item \href{https://en.wikipedia.org/wiki/ZEPLIN-III}{ZEPLIN-III dark matter experiment}, 

\textbf{Pagina del proyecto :} \url{https://zeplin.io/}

\item \href{https://en.wikipedia.org/wiki/UK_Dark_Matter_Collaboration}{\textbf{UKDMC} (\textbf{UK} \textbf{D}ark \textbf{M}atter \textbf{C}ollaboration)}, 

\textbf{Pagina del proyecto :} \url{https://hepwww.pp.rl.ac.uk/groups/ukdmc/ukdmc.html}
\end{itemize_f}
\end{itemize_f}


\subsubsection{Forma indirecta}
Otro mecanismo de investigación es cuando el valor de la propiedad física se obtiene a partir de lecturas directas de otras propiedades y de una expresión matemática que las relacione. Las medidas indirectas calculan el valor de la medida mediante una expresión matemática fundamentada por la teoría, previo cálculo de las magnitudes que intervienen en la expresión por medidas directas. Algunas investigaciones relacionadas con este mecanismo son: 

\begin{itemize_f}

\item[-] \href{https://en.wikipedia.org/wiki/Alpha_Magnetic_Spectrometer}{\textbf{AMS-02} (\textbf{A}lpha \textbf{M}agnetic \textbf{S}pectrometer)}
\begin{itemize_f}\label{AMS}
\item \textbf{Nombre:} Espectrómetro Magnético Alfa
\item \textbf{Resumen:} Busca con un detector localizado en Estación Espacial Internacional o \textbf{ISS} (\textbf{I}nternational \textbf{S}pace \textbf{S}tation) medir la antimateria en los rayos cósmicos, detectando picos en el flujo de positrones, antiprotones o rayos gamma pudiendo indicar la presencia de \neutralinos. El \textbf{AMS-01} es referido al prototipo de \textbf{AMS}, conteniendo este una versión simplificada del detector usado. Algunos de sus resultados se muestran en las referencias \cite{li_antiproton_2017,battiston_anti_2008}
\item \textbf{Pagina del proyecto :} \url{https://ams.nasa.gov/}
\end{itemize_f}

\item[-] \href{https://en.wikipedia.org/wiki/ANTARES_(telescope)}{\textbf{ANTARES} (\textbf{A}stronomy with a \textbf{N}eutrino \textbf{T}elescope and \textbf{A}byss environmental \textbf{RES}earch project)}
\begin{itemize_f}\label{antares}
\item \textbf{Nombre:} Astronomía con un Proyecto de Investigación Ambiental del Telescopio de Neutrinos y Abyss.
\item \textbf{Resumen:} Busca con sus tubos fotomultiplicadores detectar la radiación Cherenkov emitida cuando el muón pasa a través del agua, las técnicas de detección utilizadas consiguen en distinguir entre la señal de muones "que van hacia arriba", de neutrinos muónicos que interaccionan antes de llegar por debajo al detector y del alto flujo de muones procedentes de la atmósfera, con los datos y la alta resolución de estos pretende buscar indicaciones de materia oscura detectando el proceso de aniquilación del neutralino en el Sol. El proyecto \textbf{ANTARES} complementa el Observatorio de Neutrinos IceCube en la Antártida. Otros telescopios de neutrinos diseñados para su uso en el área cercana incluyen el telescopio griego \textbf{NESTOR} y el italiano \textbf{NEMO}. 

\item \textbf{Pagina del proyecto :} \url{https://antares.in2p3.fr/}

~~~~~~~~~~~~~~~~~~~~~~~~~~~~~~~~~\url{https://icecube.wisc.edu/}

~~~~~~~~~~~~~~~~~~~~~~~~~~~~~~~~~\url{https://cds.cern.ch/record/5841}

~~~~~~~~~~~~~~~~~~~~~~~~~~~~~~~~~\url{http://nemo.in2p3.fr/nemow3/index.html}
\end{itemize_f}

\item[-] \href{https://en.wikipedia.org/wiki/Calorimetric_Electron_Telescope}{\textbf{CALET} (\textbf{CAL}orimetric \textbf{E}lectron \textbf{T}elescope)}
\begin{itemize_f}\label{calet}
\item \textbf{Nombre:} Telescopio de electrones calorimétrico
\item \textbf{Resumen:} Busca realizar un seguimiento de la trayectoria de electrones, protones, núcleos y rayos gamma, mediante la medición de su dirección, carga y energía, para esto hace uso de un telescopio espacial de alta precisión.
\item \textbf{Pagina del proyecto :} \url{https://iss.jaxa.jp/en/kiboexp/ef/calet/}
\end{itemize_f}

%\item[-] \href{https://en.wikipedia.org/wiki/CERN_Axion_Solar_Telescope}{\textbf{CAST} (\textbf{C}ERN \textbf{A}xion \textbf{S}olar \textbf{T}elescope) }
%\begin{itemize_f}
%\item \textbf{Nombre:} 
%\item \textbf{Resumen:} es un experimento en física de astropartículas para buscar axiones que se originan en el sol. El experimento, realizado en el CERN en Suiza, comenzó en 2002 con la primera toma de datos a partir de mayo de 2003. La detección exitosa de los axiones solares constituiría un descubrimiento importante en la física de partículas y también abriría una nueva ventana sobre La astrofísica del núcleo solar.
%\item \textbf{Pagina del proyecto :} \url{}
%\end{itemize_f}

\item[-] \href{https://en.wikipedia.org/wiki/Dark_Matter_Particle_Explorer}{\textbf{DAMPE} (\textbf{DA}rk \textbf{M}atter \textbf{P}article \textbf{E}xplorer)}
\begin{itemize_f}
\item \textbf{Nombre:} Explorando Particulas de Materia Oscura
\item \textbf{Resumen:} Busca señal de descomposición indirecta de un hipotético candidato de materia oscura \WIMP~  mediante la detección rayos gamma de alta energía, electrones e iones de rayos cósmicos, para esto se hace uso de un telescopio espacial localizado en el satélite \textbf{CAS}.
\item \textbf{Pagina del proyecto :} \url{http://dpnc.unige.ch/dampe/}
\end{itemize_f}

\item[-] \href{https://en.wikipedia.org/wiki/Fermi_Gamma-ray_Space_Telescope}{\textbf{FGST} (\textbf{F}ermi \textbf{G}amma-ray \textbf{S}pace \textbf{T}elescope)}
\begin{itemize_f}\label{fermi}
\item \textbf{Nombre:} Telescopio Espacial de Area Grande de Rayos Gamma  
\item \textbf{Resumen:} Busca haciendo haciendo uso de un observatorio espacial muestras astronómicas de rayos gamma desde la órbita terrestre baja para para estudiar fenómenos astrofísicos y cosmológicos como núcleos galácticos activos, púlsares, otras fuentes de alta energía y materia oscura. Su instrumento principal es el Telescopio de Área Grande o \textbf{LAT} (\textbf{L}arge \textbf{A}rea \textbf{T}elescope), con el cual los astrónomos pretenden realizar un levantamiento de todo el cielo. %El proyecto es un experimento reconocido del CERN.
\item \textbf{Pagina del proyecto :} \url{https://glast.sites.stanford.edu/}
\end{itemize_f}

\item[-] \href{https://en.wikipedia.org/wiki/PAMELA_detector}{\textbf{PAMELA} (\textbf{P}ayload for \textbf{A}ntimatter \textbf{M}atter \textbf{E}xploration and \textbf{L}ight-nuclei \textbf{A}strophysics)}
\begin{itemize_f}\label{pamela}
\item \textbf{Nombre:} Exploración de la Materia-Antimateria y Astrofísica de los Núcleos de Luz.
\item \textbf{Resumen:} Busca estudiar y detectar rayos cósmicos, con un enfoque particular en su componente antimateria, en forma de positrones y antiprotones, además monitorea a largo plazo de la modulación solar de los rayos cósmicos, partículas energéticas del Sol, partículas de alta energía en la magnetosfera de la Tierra y electrones jovianos, con el objetivo de detectar evidencia de aniquilación de materia oscura.
\item \textbf{Pagina del proyecto :} \url{https://pamela.roma2.infn.it/}
\end{itemize_f}

\item[-] \href{https://stratocat.com.ar/fichas-e/1991/FSU-19910923.htm}{\textbf{MASS} (\textbf{M}atter \textbf{A}ntimatter \textbf{S}uperconducting \textbf{S}pectrometer)}
\begin{itemize_f}\label{mass}
\item \textbf{Nombre:} Espectrómetro Superconductor de Materia-Antimateria.
\item \textbf{Resumen:} Busca con la adaptación de la configuración básica de la Instalación de Imanes en Globo investigar partículas de alta energía usando un espectrómetro de imán superconductor, un dispositivo de tiempo de vuelo, un contador de gas cherenkov y un calorímetro de imagen de tubo streamer, de esta manera medir antiprotones en el rango de energías entre $4-20~GeV$ y positrones de aproximadamente $4-10~GeV$. Se utilizó la misma configuración del experimento \textbf{MASS-1} excepto por el sistema de seguimiento.

\item \textbf{Pagina del proyecto :} \url{https://stratocat.com.ar/fichas-e/1991/FSU-19910923.htm}
\end{itemize_f}

\item[-] \href{https://core.ac.uk/display/25103181}{\textbf{CAPRICE} (\textbf{C}osmic \textbf{A}nti\textbf{P}article \textbf{R}ing \textbf{I}maging \textbf{C}herenkov \textbf{E}xperiment)}
\begin{itemize_f}
\item \textbf{Nombre:} Experimento Cósmico de Imágenes de Anillo de Antipartículas de Cherenkov.
\item \textbf{Resumen:} Busca estudiar el flujo de rayos cósmicos sin demasiado fondo de partículas producidas atmosféricamente, esto es posible por el uso de un espectrómetro capaz de discriminar entre diferentes partículas. El proyecto se enfoca en estudiar los núcleos de antimateria, luz en los rayos cósmicos así como los muones en la atmósfera, específicamente mide el flujo de las antipartículas (antiprotones y positrones) por encima de aproximadamente $5~ GeV$ y relaciona los flujos con modelos que incluyen la producción exótica de antipartículas como partículas supersimétricas de materia oscura. %El flujo de muones se mide durante el descenso del globo a través de la atmósfera. Estas mediciones de flujo son importantes para los cálculos del flujo de neutrinos atmosféricos y, por lo tanto, para la interpretación de la anomalía del neutrino atmosférico.
\item \textbf{Pagina del proyecto :} \url{https://cds.cern.ch/record/5608}
\end{itemize_f}

\item[-] \href{}{\textbf{HEAT} (\textbf{H}igh-\textbf{E}nergy \textbf{A}ntimatter \textbf{T}elescope)}
\begin{itemize_f}
\item \textbf{Nombre:} Telescopio de Antimateria de Altas Energías
\item \textbf{Resumen:} Busca optimizadar la detección e identificación de electrones de rayos cósmicos y positrones a energías de aproximadamente $1~ GeV$ hasta $50~GeV$, mediante la implementación de un imán superconductor de dos bobinas y un hodoscopio de seguimiento de precisión, complementado con un sistema de tiempo de vuelo, un detector de radiación de transición y un contador de ducha electromagnético, de esta forma medir la diferencia en el tiempo entre la detección de una partícula ionizante en un tubo de deriva y un impulso generado por el disparador del experimento. Algunos de sus resultados se muestran en la referencia \cite{hooper_kaluza-klein_2004}.
\item \textbf{Pagina del proyecto :} \url{http://stratocat.com.ar/fichas-e/1994/FSU-19940503.htm}
\end{itemize_f}

\item[-] \href{https://en.wikipedia.org/wiki/Large_Hadron_Collider}{\textbf{LHC} (\textbf{L}arge \textbf{H}adron \textbf{C}ollider)}
\begin{itemize_f}\label{lhc}
\item \textbf{Nombre:} Gran Colisionador de Hadrones
\item \textbf{Resumen:} Ya que debido a que una partícula de materia oscura debería tener interacciones insignificantes con la materia visible normal, entonces estás interacciones pueden detectarse indirectamente como energía y momento faltantes que escapan de los detectores como resultado de las colisiones de haces de protones. Cualquier descubrimiento en las búsquedas de los colisionadores debe ser corroborado por resultados en los sectores de detección indirecta o directa en otros experimentos.
\item \textbf{Pagina del proyecto :} \href{https://home.cern/science/accelerators/large-hadron-collider}{\texttt{https://home.cern/science/accelerators/}}

~~~~~~~~~~~~~~~~~~~~~~~~~~~~~~~~~\href{https://home.cern/science/accelerators/large-hadron-collider}{ \texttt{large-hadron-collider}}. %aaaaaaaaaaaaaaa aaaaaaaaaaaaaaa \url{https://wikimili.com/en/China_Jinping_Underground_Laboratory}
\end{itemize_f}


\item[-] Otros experimentos : 
\begin{itemize_f}
\item[-] \href{https://en.wikipedia.org/wiki/Microlensing_Observations_in_Astrophysics}{\textbf{MOA} (\textbf{M}icrolensing \textbf{O}bservations in \textbf{A}strophysics) }

\textbf{Pagina del proyecto :} \url{http://www.tekapotourism.co.nz/info/mt_john.html}

\item[-] \href{https://en.wikipedia.org/wiki/VERITAS}{\textbf{VERITAS} (\textbf{V}ery \textbf{E}nergetic \textbf{R}adiation \textbf{I}maging \textbf{T}elescope \textbf{A}rray \textbf{S}ystem)}

\textbf{Pagina del proyecto :} \url{https://veritas.sao.arizona.edu/}
\end{itemize_f}


\end{itemize_f}

%\subsubsection{Generanción de materia oscura por coalición}


